\section{Part 2 Task 1}

\subsection{Simple Introduction}

In the second part of this assignment you will implement a GAN (Generative Adversarial Network) \cite{goodfellow2014generative} that generates images similar to those of the training set.
The training set is MNIST handwritten digit database.

\begin{figure}[!htbp]
  \centering
  \includegraphics[width=1\textwidth]{img/p2t1/GAN_struct.png}
  \caption{A Typical GAN Structure}
  \label{fig:p2_gan_struct}
\end{figure}

\subsection{Function Analyze}

GAN is a neural network structure composed of a generator and a discriminator, designed to generate realistic data samples, such as images, text, or audio.
I show the overview of the structure and training methods:

\begin{itemize}
  \item {
      Network Structure
      \begin{itemize}
        \item {Generator:
            The generator accepts a random noise vector as input and outputs new samples similar to the training data.
            It usually consists of a network that turns low-dimensional noise vectors to high-dimensional data space.
          }
        \item {Discriminator:
            The discriminator accepts real samples and fake samples generated by the generator as input and tries to distinguish them.
            It usually consists of a convolutional network that is used to classify whether the input sample is a real sample or a fake sample generated by the generator.
          }
      \end{itemize}
    }
  \item {
      Training Method

      During the training process, the generator and discriminator compete with each other and improve each other's performance through adversarial training.

      The training process is usually divided into two stages: first, the generator generates fake samples, and then the discriminator evaluates the authenticity of these fake samples.
      The feedback given by the discriminator is then used to update the parameters of the generator and discriminator.

      The goal of the generator is to generate samples that are as realistic as possible to fool the discriminator into thinking the generated samples are real.
      And the goal of the discriminator is to distinguish between real samples and fake samples generated by the generator as much as possible, so as to effectively detect fake samples generated by the generator.
    }
\end{itemize}

\subsection{Structure Analysis}

Training the GAN involves playing a minmax game between the generator and discriminator.
In other words, our optimization objective is

\begin{equation}
  \min\limits_{G} \max\limits_{D}V(D,G)=
  \min\limits_{G} \max\limits_{D}\mathbb{E}_{p}[\log{D(X)}] + \mathbb{E}_{q}[\log{(1-D(G(Z)))}]
\end{equation}

\subsubsection{Generator Layers}

\begin{enumerate}
  \item $100$ input, \texttt{latent\_dim}
  \item $100 \to 128 $ Linear Layer
  \item LeakyRelu
  \item $128 \to 256 $ Linear Layer
  \item BatchNorm, LeakyReLU
  \item $256 \to 512 $ Linear Layer
  \item BatchNorm, LeakyReLU
  \item $512 \to 1024$ Linear Layer
  \item BatchNorm, LeakyReLU
  \item $1024 \to 768$ Linear Layer
  \item $768$ output, image
\end{enumerate}

\subsubsection{Discriminator Layers}

\begin{enumerate}
  \item $784$ input, a image
  \item $784 \to 512 $ Linear Layer
  \item LeakyReLU
  \item $512 \to 256 $ Linear Layer
  \item LeakyReLU
  \item $256 \to 1 $ Linear Layer
  \item $1$ output, means real or fake
\end{enumerate}
