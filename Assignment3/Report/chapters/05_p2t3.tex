

\section{Part 2 Task 3}

\subsection{Simple Introduction}

In this part, I need to sample 2 images from my GAN. Interpolate between these two digits in latent space and include the results in my jupyter notebook.
Use 7 interpolation steps, resulting in 9 images (including start and end point).

\subsection{Generate Strategy Analysis}

\subsubsection{Strategy Introduction}

In this part of the implementation, I first randomly generate two noises with length \texttt{latent\_dim} as the begin and end noise,
then interpolate 7 noises averagely between the two noises, and finally generate the results through the Generator.

\subsubsection{Set notations}

\begin{itemize}
  \item $k$, \texttt{steps}, $9$ total steps to generate
  \item $n_1$, \texttt{noise\_begin}, begin noise
  \item $n_k$, \texttt{noise\_end}, end noise
\end{itemize}

\subsubsection{Define Expression}

\begin{align}
  \alpha_i &= \frac{i-1}{k} (1 \leq i \leq k) \\
  n_i &= (1 - \alpha_i) \cdot n_1 +\alpha_i \cdot n_k
\end{align}

Then the generator use noise $n_1 \ldots n_k$ to generate $k$ images respectively.

\subsection{Result Visualization}

Figure \ref{fig:p2t3_results} shows some results.

\begin{figure}[!htbp]
  \centering
  \begin{subfigure}[b]{0.85\textwidth}
    \includegraphics[width=\textwidth]{img/p2t3/whole1.png}
    \caption{Case 1 (7 to 8)}
  \end{subfigure}
  \begin{subfigure}[b]{0.85\textwidth}
    \includegraphics[width=\textwidth]{img/p2t3/whole2.png}
    \caption{Case 2 (1 to 7)}
  \end{subfigure}
  \begin{subfigure}[b]{0.85\textwidth}
    \includegraphics[width=\textwidth]{img/p2t3/whole3.png}
    \caption{Case 3 (0 to 5)}
  \end{subfigure}
  \caption{Task 3 Sample Results}
  \label{fig:p2t3_results}
\end{figure}

\subsection{Result Analysis}

\subsubsection{Generate Analysis}

We can see that if the first and last noises correspond to two different digital shapes, then the value in the middle will be generated as the shape between the two, showing a transition-like behavior.

In the process of generating images, the interpolation noise generated by my strategy may also be randomly generated directly. In this case, such a image looks like a number and another number.
But the number image do not directly exist in the MNIST data set which is used for training. As a result, GAN is like generating a non-existent image  out of thin air.

\subsection{Application Analysis}

With this extension of Assignment, we can imagine that GAN can be applied to generate realistic world images, including human faces, animals, natural scenery, etc.
It can even learn a certain artistic style or type of pictures to generate new pictures similar to this style.
In this way, GAN can also be used as an implementation method for the recently popular AI painting.

But this ``learning'' is also controversial.
MNIST in this assignment is an academic open source data set, and it is understandable to use it for GAN training.
For works of an artist with obvious style, or facial portraits, will the copyright or portrait rights of others be infringed during the learning process?

Although these problems are not directly related to the academic content of deep learning, it is also a point that has to make people think deeply.
