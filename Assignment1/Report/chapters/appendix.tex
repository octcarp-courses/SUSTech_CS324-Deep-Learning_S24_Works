\begin{appendix}

  \section{Extra Pictures}

  \begin{figure}[!ht]
    \centering
    \begin{subfigure}[b]{0.5\textwidth}
      \includegraphics[width=\textwidth]{img/Part1/test2_data.png}
      \caption{Test case 2 Data}
    \end{subfigure}
    \begin{subfigure}[b]{1\textwidth}
      \includegraphics[width=\textwidth]{img/Part1/test2_curve.png}
      \caption{Test case 2 Accuracy and Loss}
    \end{subfigure}
    \caption{Part 1 Test Case 2}
    \label{fig:p1test2}
  \end{figure}

  \begin{figure}[!ht]
    \centering
    \begin{subfigure}[b]{0.5\textwidth}
      \includegraphics[width=\textwidth]{img/Part1/test3_data.png}
      \caption{Test case 3 Data}
    \end{subfigure}
    \begin{subfigure}[b]{1\textwidth}
      \includegraphics[width=\textwidth]{img/Part1/test3_curve.png}
      \caption{Test case 3 Accuracy and Loss}
    \end{subfigure}
    \caption{Part 1 Test Case 3}
    \label{fig:p1test3}
  \end{figure}

  \begin{figure}[!htbp]
    \centering
    \includegraphics[width=0.9\textwidth]{img/Part2/curve_batch_-1.png}
    \caption{Mini Batch Size 1 with 1500 Iterations}
    \label{fig:p2batch-1}
  \end{figure}


  \begin{figure}[!htbp]
    \centering
    \includegraphics[width=0.9\textwidth]{img/Part2/curve_batch_-10.png}
    \caption{Mini Batch Size 10 with 1500 Iterations}
    \label{fig:p2batch-10}
  \end{figure}


  \begin{figure}[!htbp]
    \centering
    \includegraphics[width=0.9\textwidth]{img/Part2/curve_batch_-100.png}
    \caption{Mini Batch Size 100 with 1500 Iterations}
    \label{fig:p2batch-100}
  \end{figure}


\end{appendix}
